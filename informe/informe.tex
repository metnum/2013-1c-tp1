\documentclass[10pt,a4paper]{article}
\usepackage[utf8]{inputenc} 
% para poder usar tildes en archivos UTF-8 
\usepackage[spanish]{babel} 
% para que comandos como \today den el resultado en castellano 
\usepackage{a4wide} 
% márgenes un poco más anchos que lo usual 
\usepackage[conEntregas]{caratula}
\usepackage{makeidx}
\usepackage{graphicx}
\usepackage{grffile}
\usepackage{amsmath}
\usepackage{amssymb}

\begin{document}

\titulo{Trabájo Práctico 1} \subtitulo{Quake III FastInvSqrt()}

\fecha{\today}

\materia{Métodos Numéricos}
% \grupo{Grupo Los Amantes de tu Hermana}

\integrante{Escalante, José}{822/06}{joe.escalante@gmail.com}
\integrante{Raskovsky, Iván Alejandro}{57/07}{iraskovsky@dc.uba.ar}
\integrante{Osinski, Andrés}{405/07}{andres.osinski@gmail.com}

% TODO: Agregar abstract y 4 palabras claves

% El tutulo debera ser breve y apropiado para una rapida identificacion del
% contenido del trabajo. El resumen, de no mas de 200 palabras, debera explicar
% brevemente el trabajo realizado y las conclusiones de los autores de manera
% que pueda ser util por ser solo para dar una idea del contenido del trabajo.
% Las palabras clave, no mas de cuatro, deben ser terminos tecnicos que den una
% idea del contenido del trabajo para facilitar su busqueda en una base de
% datos tematica.

\maketitle
\tableofcontents
\newpage

    \section{Abstract}

En este trabajo vamos a mostrar dos formas de poder
calcular la inversa de la raíz cuadrada, a partir de adaptar el problema a la
búsqueda de ceros de una función. Iremos documentando también los resultados
obtenidos a partir de ciertas familias de inputs elegidas con cierto
criterio.\\

En particular veremos el método de Newton y el de la Secante para llegar al
resultado mencionado.\\

Por último, a partir de la observación de los experimentos daremos nuestro juicio sobre las ventajas y desventajas de ambos métodos y nuestra recomendación sobre cuál usar.\\

{\bf Palabras clave:}
\begin{itemize} 
    \item Método de Newton 
    \item Método de la Secante 
    \item Ceros de funciones
\end{itemize}

    \newpage
    \section{Introducción Teórica}

Las funciones que usaremos para aproximar la inversa de la raíz son las siguientes:

\begin{displaymath}
    f(x) = x^2 - \alpha
\end{displaymath}

\begin{displaymath}
    e(x) = \frac{1}{x^2} - \alpha
\end{displaymath}

A partir de este momento nos enfocaremos en encontrar los ceros de estas dos funciones usando los métodos de Newton y de la Secante:

\begin{displaymath}
    x_{n + 1} = x_n - \frac{h(x_n)}{h'(x_n)}
\end{displaymath}

\begin{displaymath}
    x_{n + 1} = x_{n - 1} - h(x_{n - 1})\frac{x_{n - 1} - x_{n - 2}}{h(x_{n - 1}) - h(x_{n - 2})}
\end{displaymath}

Donde $\displaystyle h(x) = f(x)$ o $\displaystyle h(x) = e(x)$

En este caso para calcular la inversa de la raíz cuadrada
($\displaystyle\frac{1}{\sqrt{\alpha}}$) buscaremos los ceros de dos funciones.

La resolución de estas 2 funciones es equivalente a resolver el problema de la
inversa de la raíz. A partir de que el problema se plantea de esa manera es que
podemos hacer uso de métodos para encontrar ceros de funciones.\\

Las funciones en cuestión son


Cada una de estas funciones provee formas de encontrar la raiz de un numero.
Veamos el caso de $f(x)$.

Los métodos elegidos para encontrar las raices de estas funciones son el método
de Newton y el método de la Secante.

\subsubsection{Análisis de las funciones}

Notemos que $\alpha$ siempre tiene que ser positivo, sino no se le puede calcular la raiz cuadrada a un numero negativo. Mas aún $\alpha > 0$ ya que sino estaríamos diviendo por cero.

\subsubsection{f(x)}

Primero demostrar que cuando $f(x) == 0 -> x == \sqrt{\alpha}$
entonces una vez encontradas las raices podemos hacer $1/x$ para encontrar
$\displaystyle\frac{1}{\sqrt{\alpha}}$.

En el caso de $\displaystyle e(x) == 0 -> x == \frac{1}/{\sqrt{\alpha}}$ por lo que no tenemos que hacer
ninguna otra cuenta.\\

Analicemos graficamente las funciones:\\

% intertart grafico lindo de f(x)

$f(x)$ es una parábola. Al ser $\alpha > 0$ podemos ver que *siempre* tiene dos
raíces. Más aún $f(x)$ es simétrica por lo cual podemos encontrar cualquiera de
las dos raíces y con esta cambiarle el signo y obtener la otra. De esta forma
no nos preocuparemos por obtener la raíz positiva ya que nos es indistinto que
raíz conseguimos con los métodos.\\

% insertar grafico lindo de e(x)

$e(x)$ con $\alpha > 0$ también tiene siempre dos raíces por lo que al igual que con
$f(x)$ nos es indistinto cuál de las dos obtenemos. En este caso es importante
notar que en el 0 hay una asíntota de las ordenadas.\\
 
Los dos métodos que elegimos trabajan con la tangente de las funciones en un
punto o con una aproximación de esta. Veamos la concavidad de las funciones.

% a partir de aca es todo chamuyo... ver que dejar y que sacar, porque en
% realida se puede hacer buen analisis analitico pero no se como.. buscar en
% google o wikipedia quizas


% graficos lindos de f''(x) y e''(x).

veamos que $f''(x) == 2$. esta funcion es convexa. al ser constante y por la
forma que tienen las derivadas asumimos que siempre va a converger.

veamos $e''(x)$ se nos va al sorete en el 0. por algo que no se, que nos tenemos
que sacar de la galera esta no siempre va a converger!!! pero no sabemos bien
por que


Cómo se puede ver los valores que hacen cero a esas funciones son de la forma
$x = \sqrt{\alpha}$ en el primer caso y $x = \frac{1}{\sqrt{\alpha}}$ en el
segundo.

\subsubsection{Métodos}
Podemos ver que se puede utilizar tanto Newton como Secante porque las
funciones cumplen lo que pida cada una.

Podemos utilizar la secante porque .. y también Newton porque pide todo esto y
que exista la derivada primera de la funcion, las cuales tenemos para $e$ y para
$f$.

\subsubsection{Hipótesis}
Creemos que Newton va a funcionar mucho mejor que Secante porque su órden de
convergencia teórico es mejor.

Por otro lado no estamos seguros de cuál de las dos funciones va a ser mejor
porque no podemos interpretar funciones y como afectan las tangentes a los
métodos.

% no se si esto viene aca o mas abajo, quizas mas abajo o arriba, pero si nos
% interesa ver los dos casos, el caso con bajo error y el caso rapido.
Por último queremos hacer un análisis para un cálculo aproximado pero MUY
RAPIDO al estilo quake. Para este puede ser que las conclusiones sean
diferentes a casos donde puede haber mas iteraciones.  por que? porque para un
caso podemos empezar mucho ma cerca de la raiz y hacer pocas iteraciones y en
el otro caso podemos empezar muy lejos pero con algunas iteraciones mas ya
estar mucho mas cerca. pero para ej 3 iteraciones es mejor el metodo y funcion
\"lento\".


% Contendra una breve explicacion de la base teorica que fundamenta los metodos
% involu- crados en el trabajo, junto con los metodos mismos. No deben
% incluirse demostraciones de propiedades ni teoremas, ejemplos innecesarios,
% ni deniciones elementales (como por ejemplo la de matriz simetrica). En vez
% de deniciones basicas es conveniente citar ejemplos de bibliografa
% adecuada.  Una cita vale mas que mil palabras
    \newpage
    \section{Desarrollo}
% Deben explicarse los metodos numericos que utilizaron y su aplicacion al
% problema concreto involucrado en el trabajo practico. Se deben mencionar los
% pasos que si- guieron para implementar los algoritmos, las dicultades que
% fueron encontrando y la descripcion de como las fueron resolviendo. Explicar
% tambien como fueron planteadas y realizadas las mediciones experimentales.
% Los ensayos fallidos, hipotesis y conjeturas equivocadas, experimentos y
% metodos malogrados deben gurar en esta seccion, con una breve explicacion
% de los motivos de estas fallas (en caso de ser conocidas)

\subsection{Implementación}

Hicimos una implementación que nos permite decidir paramétricamente lo siguiente:

\begin{itemize}
    \item Función $e$ o $f$
    \item Método Newton o Secante
    \item Criterio de parada
    \begin{itemize}
        \item Iteraciones
        \item Tiempo
        \item Error Absoluto
        \item Error Relativo
    \end{itemize}
    \item Eleccion de $x_0$ y $x_1$ ($x_1$ solo para la Secante)
\end{itemize}

Esto lo desarrollamos en \verb|C| y luego hicimos unos $helpers$ en Python para poder
crear experimentos y analizar los resultados. Cada corrida nos devuelve todos
los valores de los criterios de parada y las aproximaciones en cada iteracion.\\

Estos experimentos son replicables y los puede ver en el anexo.\\

El codigo se puede optimizar MUCHO, sobre todo en las cuentas que realiza el
cuerpo de cada metodo. Ahora hay una implementacion naif que replica el enfoque
analitico pero que a nivel fierro se puede mejorar.\\

Por otro lado, agregamos mucho overhead con la flexibilidad, pero podemos ver
que esto afecta a los dos metodos y funciones por igual por lo que no deberia
afectar las comparaciones y resultados.

\subsubsection{Métodos}
Tanto el método de Newton como el de Secante se pueden utilizar para resolver ambas funciones,
en particular porque, para Newton, existen las derivada primeras de las funciones.

\subsection{Convergencia}
empezamos con x0 fijos y vimos que tanto newton como secante siempre convergian
para f(x). % mentira, hacer alguas corridas, tengan fe Esto es tanto para
alphas muy chicos (cercanos al cero) como para alhpas muy grandes.

en cambio para e(x) vimos que nunca convergia, al menos las aproximaciones
crecian muy rapidamente, tanto que en un puñado de iteraciones superaban el
mayor número representable por punto flotante doble (devolvia inf). % mostrar
alguna corrida donde pase esto.

Entonces para f(x) decidimos utilizar un x0=alpha que siempre funciona. Newton
tiene una convergencia teorica muy rapida por lo que no deberia influir.

En el caso de e(x) es muy interesante notar como las aproximaciones se alternan
entre positivas y negativas cuando no converge. esto se debe a ???.

% todavia no encontramos un x0 y x1 para e(x) para que converja

empiricamente (diseñar y pensar exp) encontramos que si elejimos un x1 y x0
cerca de e(x) converge pero sino no.

un primer acercamiento a x0 que se nos ocurrio fue utilizar 0.0alpha y
0.00alpha. Esto funciona hasta alphas no muy grandes (mas o menos hasta $10^4$,
ajustar es numero con un experimento).

% explicar que es DBL EPSILON y que es DBL MIN
Al hacer estas pruebas nos parece que si x0 y x1 son menos que la raiz
converge. Entonce decidimos utilizar $DBL_MIN$ y $DBL_MIN * 2$ pero estas rompian
las operaciones aritmeticas de punto flotante. Entonce pasamos a utilizar
$DBL_EPSILON$ y encontramos que funciona muy bien tanto para numeros muy chicos
como para numeros muy grandes. El problema que tiene es que requiere de muchas
iteraciones.

Grafiquemos la cantidad de iteraciones necesarias para diferentes alphas (muy
chicos y cercanos al $1$ y muy grandes) para $DBL_EPSILON$. Lo podemos mejorar?
creemos que si, pero quedará para futura investigacion.

Otra cosa que se nos ocurrio es utilizar aproximaciones de f(x) primero hasta
un cierto error grosero y luego continuar refinando con e(x). Pero para que
esto tenga sentido tuvimos que hacer una comparacion entre los metodos. Ver mas
abajo.


    \newpage
    \section{Resultados}
% Deben incluir los resultados de los experimentos, utilizando el formato mas
% adecuado para su presentacion. Deberan especicar claramente a que
% experiencia corresponde cada resultado. No se incluiran aqu corridas de
% maquina. Algo fundamental en su aprendizaje en la materia es la presentacion
% de resultados de forma clara y concisa para el lector

blah blah blah cual es mejor con cual nos quedamos para el caso preciso y con
cual para el caso rapido donde rapido puede ser 1 2 o 3 iteraciones y preciso
cientos.

% Ver un grafico que muestra los picos del error relativo conforme el alapha se aleja del x0. La explicacion es que conforme mas lejos esta el x0 del resultado, tiene que realizar mas itreraciones y el error relativo de la ultima iteracion que vale crece conforme aumenta la distancia entre alpah y x0 hasta que la iteracion no califica para entrar en el criterio de tension relativo. Entonces debe realizar una iteracion mas que resulta mas precisa y asi sucesivamente

% Hay que un hacer un grafico del mismo experimento pero con el numero de iteraciones y queremos mosrtar que conforme la distancaio del x0 al alpha crece pasa que la cant de iteraciones para converger aumenta.

% Hacer un exp de un alpha fijo con un x0 que se va corriendo. Para observar las iteraciones sobre un valor fijo con la distancia entre x0 y alpha, corriendo el x0. Ademas un grafico de tiempo.

% Un grafico de newton de 1 al 10000 con error absoluto con x0 cercanos al resultado real y compararlo con las iteraciones del relativo

Los tipos de experimentos:
* Comparar newton y secante para f(x) y e(x) en función de:
    1. cantidad de iteraciones
    2. tiempo de ejecución
    3. error relativo, error absoluto
    4. orden de convergencia
* Los tipos de inputs a utilizar pueden ser:
    1. crecientes de 1 a 1000 en intervalos de 1, 5 y 10
    2. crecientes de 0 a 1 en intervalos chiquitos, la idea es ver cuando forzamos a que la precisión se pierda
    2. aleatorios con magnitud definida
    \newpage
    \section{Discusión y Conclusiones}
% Se incluira aqu un analisis de los resultados obtenidos en la seccion
% anterior (se analizara su validez, coherencia, etc.). Deben analizarse como
% materianimo los lostems pedidos en el enunciado. No es aceptable decir que
% \los resultados fueron los esperados", sin hacer clara referencia a la
% teoremasa a la cual se ajustan. Ademas, se deben mencionar los resul- tados
% interesantes y los casos \patologicos" encontrados.

En esta parte analizaremos, cuestionaremos y concluiremos sobre aquellos puntos que nos parecieron interesantes pero que además fueron preponderantes en el desarrollo del presente trabajo.

\subsection{Elección de los puntos iniciales de las sucesiones} % (fold)
\label{sub:elecci_n_de_los_puntos_iniciales_de_las_sucesiones}

En la secciones \ref{ssub:ajuste_f_x0_newton}, \ref{ssub:ajuste_e_x0_newton}, \ref{ssub:ajuste_e_x0_x1_secante}
% Hablar sobre las variantes que descartamos al hacer este tipo de enfoque y sus posibles repercusiones

% subsection elecci_n_de_los_puntos_iniciales_de_las_sucesiones (end)

\subsection{Análisis de los resultados en función del tiempo y las iteraciones} % (fold)
\label{sub:an_lisis_de_los_resultados_en_funci_n_del_tiempo_y_las_iteraciones}

\subsubsection{Iteraciones} % (fold)
\label{ssub:iteraciones}

Para este gráfico (Figura~\ref{fig:comparacion_iteraciones}) no existe demasiada discusión ya que el resultado cumple con las expectativas que teníamos previamente. Cabe destacar que las pendientes de las rectas con respecto al origen en módulo no son iguales, esto es interesante dado que el análisis previo de cada función nos mostró que podíamos hacer uso indistinto de cualquiera de las raíces, lo cual hacía suponer que el gráfico tendría que tener mayor simetría.

Mas allá de estos detalles, pudimos cerciorar empíricamente que si se toma en cuenta la cantidad de iteraciones, usando $f(x)$ con el método de Newton es la opción mas recomendable

% subsubsection iteraciones (end)

\subsubsection{Tiempo de ejecución} % (fold)
\label{ssub:tiempo_de_ejecuci_n}

En esta gráfico (Figura~\ref{fig:comparacion_tiempos}) es en donde pudimos ver un comportamiento al menos no esperado de nuestra parte. Suponemos que nuestra hipótesis no se cumple, pues la derivada en este caso es fácil de calcular y evaluar, haciendo que el factor que menos beneficiaba a este método no afecte para este caso a la performance.

% subsubsection tiempo_de_ejecuci_n (end)

% subsection an_lisis_de_los_resultados_en_funci_n_del_tiempo_y_las_iteraciones (end)

\subsection{Resoluciones finales} % (fold)
\label{sub:resoluciones_finales}

% subsection resoluciones_finales (end)

\subsection{Trabajos futuros} % (fold)
\label{sub:trabajos_futuros}

% subsection trabajos_futuros (end)

    \newpage
    \section{Conclusiones}
% Esta seccion debe contener las conclusiones generales del trabajo. Se deben
% mencionar las relaciones de la discusion sobre las que se tiene certeza,
% junto con comentarios y observaciones generales aplicables a todo el proceso.
% Mencionar tambien posibles extensiones a los metodos, experimentos que hayan
% quedado pendientes, etc

todo depende, de que depende? de si buscas velocidad o precision

    \newpage

    \appendix
% En el apendice A se incluira el enunciado del TP. En el apendice B se
% incluiran los codigos fuente de las funciones relevantes desde el punto de
% vista numerico. Resultados que valga la pena mencionar en el trabajo pero que
% sean demasiado especicos para aparecer en el cuerpo principal del trabajo
% podran mencionarse en sucesivos apendices rotulados con las letras mayusculas
% del alfabeto romano. Por ejemplo: la demostracion de una propiedad que
% aplican para optimizar el algoritmo que programaron para resolver un
% problema.
%    \input{archivos/enunciado.tex}
    \section{Referencias}
% Es importante incluir referencias a libros, articleculos y paginas de
% Internet consultados durante el desarrollo del trabajo, haciendo referencia a
% estos materiales a lo largo del informe. Se deben citar tambien las
% comunicaciones personales con otros grupos

Wikipedia
Burden

    \section{Anexos}
% \subsection{Pseudocodigo de metodos}
% copiar pseudocodigo (comentarios) de %newton_start newton_iter secante_start
% %secante_iter y loop principal

% \subsection{Detalle de experimentos}

% en este anexo detallamos los diferentes experimentos, como replicarlos y la
% motivacion detras de cada uno de ellos.

% por cada experimento indicar en que archivo esta, y que corridas se hicieron
% con que datos y por que elegimos hacer ese exp.
\subsubsection{Enunciado}
\begin{center}
\begin{tabular}{r|cr}
 \begin{tabular}{c}
{\large\bf\textsf{\ M\'etodos Num\'ericos\ }}\\ 
Segundo Cuatrimestre 2013\\
{\bf Trabajo Pr\'actico 1}
\end{tabular} &
% \begin{tabular}{@{} p{1.6cm} @{}}
% \includegraphics[width=1.6cm]{../../logodpt.jpg}
% \end{tabular} &
\begin{tabular}{l @{}}
 \emph{Departamento de Computaci\'on} \\
 \emph{Facultad de Ciencias Exactas y Naturales} \\
 \emph{Universidad de Buenos Aires} \\
\end{tabular} 
\end{tabular}
\end{center}

\vskip 25pt
\hrule
\vskip 11pt
 
\textbf{Introducci\'on}

En las \'ultimas dos d\'ecadas se han producido avances muy significativos en el \'area de Computaci\'on Gr\'afica, en
particular en el desarrollo de animaciones y video juegos en 3D donde se obtienen resultados muy detallados y con un
alto nivel de realismo. 

Un aspecto importante que contribuye en este sentido es la iluminaci\'on de la escena y el reflejo de la luz en los
objetos o personajes de la misma. A grandes rasgos, el manejo de la iluminaci\'on se hace de la siguiente forma. Dada
una superficie en el espacio, se calculan los vectores normales a la misma (recordar \emph{An\'alisis II}) sobre un
conjunto determinado de puntos y luego estos vectores son utilizados, en conjunto con el modelo de iluminaci\'on, para
calcular su color final y la interacci\'on con otras superficies. Adem\'as, por cuestiones pr\'acticas, los vectores
normales deber ser almacenados como vectores unitarios, es decir, que su norma Euclideana ($\|~\|_2$) sea 1.

Dado un vector $y \in \mathbb{R}^3$ cualquiera, podemos convertirlo en uno unitario dividi\'endolo por $\|y\|_2$, es
decir, 
\begin{eqnarray*}
\| y \|_2 & = & \sqrt{y_1^2 + y_2^2 + y_3^2}\\
z & = & \frac{y}{\|y\|_2}.
\end{eqnarray*}
\noindent Durante la ejecuci\'on del programa, esta operaci\'on es realizada millones de veces por segundo, por lo cual
es importante realizarla en el menor tiempo posible, eventualmente resignando precisi\'on en el
resultado. Dado el contexto, peque\~nas reducciones en el tiempo de ejecuci\'on pueden mejorar considerablemente el
comportamiento general.

\textbf{El problema}

Sean $y = (y_1, y_2, y_3) \in \mathbb{R}^3$ un vector gen\'erico y $\alpha = y_1^2 + y_2^2 + y_3^2$. El problema de
obtener un vector unitario que posea la misma direcci\'on que $y$ lo podemos plantear como multiplicarlo por
$1/\sqrt{\alpha}$. Luego, el problema de normalizar un vector radica principalmente en el c\'alculo de este valor, que
involucra una divisi\'on y el c\'alculo de una ra\'iz cuadrada. 

Sin utilizar funciones ya provistas por el lenguaje de programaci\'on a utilizar, el c\'alculo de $1/\sqrt{\alpha}$ se
pueden formular como un problema de b\'usqueda de ceros de una funci\'on de (al menos) las siguientes dos formas:
\begin{itemize}
\item Aproximar $\beta = \sqrt{\alpha}$ como un cero de $f(x) = x^2 - \alpha$, y luego realizar $1/\beta$. 
\item Definir la funci\'on $e(x) = \frac{1}{x^2} - \alpha$, que permite calcular el error de una aproximaci\'on de
$1/\sqrt{\alpha}$. En particular, uno de los ceros de esta funci\'on es el valor buscado.
\end{itemize}

Estas dos reformulaciones del problema nos permiten atacarlo con m\'etodos de b\'usqueda de ceros de funciones en una
variable.

\textbf{Enunciado}

El objetivo del trabajo pr\'actico consiste en implementar un programa que permita calcular, dado $\alpha \in
\mathbb{R}$, $1/\sqrt{\alpha}$. Para ello, se deber\'a considerar las funciones $f(x)$ y $e(x)$ definidas
anteriormente, distintos m\'etodos vistos en clase que permitan resolver el problema planteado y realizar un an\'alisis
completo del comportamiento de los mismos. 

Los requisitos m\'inimos a cumplir son los siguientes:

\begin{itemize}
\item Implementar el m\'etodo de Newton para la funci\'on $f(x)$. Incluir en el informe la demostraci\'on de
convergencia (Ejercicio 3, Pr\'actica 1). Para la funci\'on $e(x)$, implementar al menos dos m\'etodos (uno de los
cuales debe ser el de Newton).   
\item Para cada m\'etodo, estudiar experimentalmente la convergencia, tiempo de ejecuci\'on, cantidad de iteraciones,
criterios de parada, precisi\'on en el resultado, y cualquier otro par\'ametro que considere necesario evaluar. Realizar experimentos
computacionales considerando un rango amplio de valores posibles para $\alpha$ y distintos puntos iniciales
para los m\'etodos. Analizar y justificar detalladamente los resultados obtenidos.
\item Una vez fijados los mejores par\'ametros para cada m\'etodo, realizar una comparaci\'on entre las tres formas
alternativas de resolver el problema (Newton para $f(x)$, y Newton m\'as el otro m\'etodo para $e(x)$) en t\'erminos de
tiempo de ejecuci\'on, precisi\'on en la soluci\'on, cantidad de iteraciones, etc. Determinar experimentalmente que
variante seleccionar\'ia para su utilizaci\'on en la pr\'actica.
\end{itemize}

\vskip 15pt

\hrule

\vskip 11pt


{\bf \underline{Fechas de entrega}}
\begin{itemize}
 \item \emph{Formato Electr\'onico:} Domingo 1° de Septiembre de 2013, hasta las 23:59 hs, enviando el trabajo (informe +
 c\'odigo) a la direcci\'on \verb+metnum.lab@gmail.com+. El subject del email debe comenzar con el texto \verb+[TP1]+
 seguido de la lista de apellidos  de los integrantes del grupo.
 \item \emph{Formato f\'isico:} Lunes 2 de Septiembre de 2013, de 17 a 18 hs.
\end{itemize}

\noindent \textbf{Importante:} El horario es estricto. Los correos recibidos despu\'es de la hora indicada ser\'an considerados re-entrega.

\end{document}
