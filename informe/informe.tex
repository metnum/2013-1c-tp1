\documentclass[10pt,a4paper]{article}
\usepackage[utf8]{inputenc} % para poder usar tildes en archivos UTF-8
\usepackage[spanish]{babel} % para que comandos como \today den el resultado en castellano
\usepackage{a4wide} % márgenes un poco más anchos que lo usual
\usepackage[conEntregas]{caratula}

\begin{document}

\titulo{Trabájo Práctico 1}
\subtitulo{Quake III FastInvSqrt()}

\fecha{\today}

\materia{Métodos Numéricos}
% \grupo{Grupo Los Amantes de tu Hermana}

\integrante{Escalante, José}{822/06}{joe.escalante@gmail.com}
\integrante{Raskovsky, Iván Alejandro}{57/07}{iraskovsky@dc.uba.ar}
\integrante{Osinski, Andrés}{003/01}{email3@dominio.com}

% TODO: Agregar abstract y 4 palabras claves

% El tutulo debera ser breve y apropiado para una rapida identicacion del contenido del
% trabajo. El resumen, de no mas de 200 palabras, debera explicar brevemente el trabajo
% realizado y las conclusiones de los autores de manera que pueda ser util por ser solo para
% dar una idea del contenido del trabajo. Las palabras clave, no mas de cuatro, deben ser
% terminos tecnicos que den una idea del contenido del trabajo para facilitar su busqueda
% en una base de datos tematica.

\maketitle

\section{Abstract}
En este trabajo vamos a mostrar distintas formas de poder calcular la inversa de la raíz cuadrada, a partir de adaptar el problema a la búsqueda de ceros de una función. Iremos documentando también los resultados obtenidos a partir de ciertas familias de inputs elegidas con cierto criterio.\\

En particular veremos el método de Newton y el de la Secante para llegar al resultado mencionado:\\

Los experimentos nos terminaron mostrando que el método de Newton termina siendo mejor en cuestiones de orden de convergencia y error.\\

{\bf Palabras clave:}
\begin{itemize}
    \item Método de Newton
    \item Método de la Secante
    \item Criterios de comparación
\end{itemize}

\newpage
\section{Introducción Teórica}
%La computación científica se caracteriza en trabajar con números reales, los cuales no son representables de manera precisa en la computadora debido a la aritmética finita que esta maneja. Teniendo esto en cuenta es que la manera estándar en cómo se calculan los resultados de ciertas operaciones puede que no sea la óptima.\\

Lo que queremos es encontrar una forma de calcular la raíz cuadrada. Hay diferentes maneras de calcular la raíz cuadrada%{link a methods square root}
En este trabajo vamos a tratar de aproximar la inversa de la raíz cuadrada a traves de diferentes metodos de ceros de funciones.

En este caso para calcular la inversa de la raíz cuadrada ($\displaystyle\frac{1}{\sqrt{\alpha}}$) buscaremos los ceros de dos funciones. La resolución de estas 2 funciones es equivalente a resolver el problema de la inversa de la raíz. A partir de que el problema se plantea de esa manera es que podemos hacer uso de métodos para encontrar ceros de funciones.\\

Las funciones en cuestión son:
\begin{displaymath}
    f(x) = x^2 - \alpha
\end{displaymath}

\begin{displaymath}
    e(x) = \frac{1}{x^2} - \alpha
\end{displaymath}

Cómo se puede ver los valores que hacen cero a esas funciones son de la forma $x = \sqrt{\alpha}$ en el primer caso y $x = \frac{1}{\sqrt{\alpha}}$ en el segundo.

% Contendra una breve explicacion de la base teorica que fundamenta los metodos
% involu- crados en el trabajo, junto con los metodos mismos. No deben
% incluirse demostraciones de propiedades ni teoremas, ejemplos innecesarios,
% ni deniciones elementales (como por ejemplo la de matriz simetrica). En vez
% de deniciones basicas es conveniente citar ejemplos de bibliografa
% adecuada.  Una cita vale mas que mil palabras

\newpage
\section{Desarrollo}
% Deben explicarse los metodos numericos que utilizaron y su aplicacion al
% problema concreto involucrado en el trabajo practico. Se deben mencionar los
% pasos que si- guieron para implementar los algoritmos, las dicultades que
% fueron encontrando y la descripcion de como las fueron resolviendo. Explicar
% tambien como fueron planteadas y realizadas las mediciones experimentales.
% Los ensayos fallidos, hipotesis y conjeturas equivocadas, experimentos y
% metodos malogrados deben gurar en esta seccion, con una breve explicacion
% de los motivos de estas fallas (en caso de ser conocidas)

\section{Resultados}
% Deben incluir los resultados de los experimentos, utilizando el formato mas
% adecuado para su presentacion. Deberan especicar claramente a que
% experiencia corresponde cada resultado. No se incluiran aqu corridas de
% maquina. Algo fundamental en su aprendizaje en la materia es la presentacion
% de resultados de forma clara y concisa para el lector

\section{Discusión}
% Se incluira aqu un analisis de los resultados obtenidos en la seccion
% anterior (se analizara su validez, coherencia, etc.). Deben analizarse como
% materianimo los lostems pedidos en el enunciado. No es aceptable decir que
% \los resultados fueron los esperados", sin hacer clara referencia a la
% teoremasa a la cual se ajustan. Ademas, se deben mencionar los resul- tados
% interesantes y los casos \patologicos" encontrados.

\section{Conclusiones}
% Esta seccion debe contener las conclusiones generales del trabajo. Se deben
% mencionar las relaciones de la discusion sobre las que se tiene certeza,
% junto con comentarios y observaciones generales aplicables a todo el proceso.
% Mencionar tambien posibles extensiones a los metodos, experimentos que hayan
% quedado pendientes, etc

\section{Apéndices}
% En el apendice A se incluira el enunciado del TP. En el apendice B se
% incluiran los codigos fuente de las funciones relevantes desde el punto de
% vista numerico. Resultados que valga la pena mencionar en el trabajo pero que
% sean demasiado especicos para aparecer en el cuerpo principal del trabajo
% podran mencionarse en sucesivos apendices rotulados con las letras mayusculas
% del alfabeto romano. Por ejemplo: la demostracion de una propiedad que
% aplican para optimizar el algoritmo que programaron para resolver un
% problema.

\section{Referencias}
% Es importante incluir referencias a libros, articleculos y paginas de
% Internet consultados durante el desarrollo del trabajo, haciendo referencia a
% estos materiales a lo largo del informe. Se deben citar tambien las
% comunicaciones personales con otros grupos

\end{document}
