\section{Discusión y Conclusiones}
% Se incluira aqu un analisis de los resultados obtenidos en la seccion
% anterior (se analizara su validez, coherencia, etc.). Deben analizarse como
% materianimo los lostems pedidos en el enunciado. No es aceptable decir que
% \los resultados fueron los esperados", sin hacer clara referencia a la
% teoremasa a la cual se ajustan. Ademas, se deben mencionar los resul- tados
% interesantes y los casos \patologicos" encontrados.

\subsection{Casos intersantes}

primer caso
===========

bin/tp1 0.0000000000000000000000000000000001 e n r 0.000000000000000000000000000001 1 2
se queda saltando entre dos numeros diferentes

con numeros tan chicos de alhpa el punto flotante explota y se pone a hacer cosas raras, analizarlo un poco mas y decir por que parece que hace esto.


segundo caso
============

una corrida, si le agregamos un 0 al error relativo corta por cantidad de iteraciones y no por llegar al error relativo

*********************** bin/tp1 9999 e s r 0.000000000000001 0.09999 0.009999
Buscando 1 / sqrt(9999.000000)
Utilizando e(x)
Utilizando Secante
Criterio de parada error relativo con limite 0.000000
Utilizando x0=0.099990 y x1=0.009999

0.010026267273090912895971982266019040253013372421264648437500000000000000000000000000000000000000000 Iter
0.010000505837327295852179354085365048376843333244323730468750000000000000000000000000000000000000000 Iter
0.010000500015077814705555248053769901162013411521911621093750000000000000000000000000000000000000000 Iter
0.010000500037503143660466697895117249572649598121643066406250000000000000000000000000000000000000000 Iter
0.010000500037503126313231938127046305453404784202575683593750000000000000000000000000000000000000000 Iter

0.010000500037503126313231938127046305453404784202575683593750000000000000000000000000000000000000000 resultado
5 iteraciones
-0.000000000001818989403545856475830078125000000000000000000000000000000000000000000000000000000000000 error absoluto
-0.000000000000000181917132067792432769858681142675088907883197372229722166281362660811282694339752197 error relativo
0.000067999999999999999463970445923166607826715335249900817871093750000000000000000000000000000000000 tiempo

*********************** bin/tp1 9999 e s r 0.0000000000000001 0.09999 0.009999
0.010000500037503124578508462150239211041480302810668945312500000000000000000000000000000000000000000 Iter
0.010000500037503124578508462150239211041480302810668945312500000000000000000000000000000000000000000 Iter
0.010000500037503124578508462150239211041480302810668945312500000000000000000000000000000000000000000 Iter
0.010000500037503124578508462150239211041480302810668945312500000000000000000000000000000000000000000 Iter
0.010000500037503124578508462150239211041480302810668945312500000000000000000000000000000000000000000 Iter
0.010000500037503124578508462150239211041480302810668945312500000000000000000000000000000000000000000 Iter

0.010000500037503124578508462150239211041480302810668945312500000000000000000000000000000000000000000 resultado
1000 iteraciones
0.000000000001818989403545856475830078125000000000000000000000000000000000000000000000000000000000000 error absoluto
0.000000000000000181917132067792432769858681142675088907883197372229722166281362660811282694339752197 error relativo
0.011587999999999999342636947119444812415167689323425292968750000000000000000000000000000000000000000 tiempo

todo depende, de que depende? de si buscas velocidad o precision
Veamos lo que podemos implicar a partir de la comparación que se hizo de cada gráfico
\begin{itemize}
\item {\bf Cantidad de iteraciones f(x) con Newton y Secante:} Acá se puede observar que a medida que el $x_0$ crece la cantidad de iteraciones crece pero mas despacio, casi logrando un crecimiento logarítmico. Sin embargo es apreciable que el método de Newton toma menos iteraciones.
\item {\bf Error Absoluto f(x) con Newton y Secante:} En los gráficos se puede apreciar que cuando $x_0$ es chico Newton y Secante alcanzan el criterio de parada casi en los mismos puntos. Pero a medida que $x_0$ crece Secante empieza a llegar antes al criterio de parada. Se podría decir que Secante es mas preciso que Newton.
\item {\bf Tiempo de Computo f(x) con Newton y Secante:} En este caso es claro ver  

\end{itemize}



