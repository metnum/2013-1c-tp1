\documentclass[10pt,a4paper]{article} \usepackage[utf8]{inputenc} % para poder
usar tildes en archivos UTF-8 \usepackage[spanish]{babel} % para que comandos
como \today den el resultado en castellano \usepackage{a4wide} % márgenes un
poco más anchos que lo usual \usepackage[conEntregas]{caratula}

\begin{document}

\titulo{Trabájo Práctico 1} \subtitulo{Quake III FastInvSqrt()}

\fecha{\today}

\materia{Métodos Numéricos}
% \grupo{Grupo Los Amantes de tu Hermana}

\integrante{Escalante, José}{822/06}{joe.escalante@gmail.com}
\integrante{Raskovsky, Iván Alejandro}{57/07}{iraskovsky@dc.uba.ar}
\integrante{Osinski, Andrés}{003/01}{email3@dominio.com}

% TODO: Agregar abstract y 4 palabras claves

% El tutulo debera ser breve y apropiado para una rapida identificacion del
% contenido del trabajo. El resumen, de no mas de 200 palabras, debera explicar
% brevemente el trabajo realizado y las conclusiones de los autores de manera
% que pueda ser util por ser solo para dar una idea del contenido del trabajo.
% Las palabras clave, no mas de cuatro, deben ser terminos tecnicos que den una
% idea del contenido del trabajo para facilitar su busqueda en una base de
% datos tematica.

\maketitle

\section{Abstract} En este trabajo vamos a mostrar distintas formas de poder
calcular la inversa de la raíz cuadrada, a partir de adaptar el problema a la
búsqueda de ceros de una función. Iremos documentando también los resultados
obtenidos a partir de ciertas familias de inputs elegidas con cierto
criterio.\\

En particular veremos el método de Newton y el de la Secante para llegar al
resultado mencionado:\\

Los experimentos nos terminaron mostrando que el método de Newton termina
siendo mejor en cuestiones de orden de convergencia y error.\\

{\bf Palabras clave:} \begin{itemize} \item Método de Newton \item Método de la
        Secante \item Criterios de comparación \end{itemize}

\newpage \section{Introducción Teórica}
%La computación científica se caracteriza en trabajar con números reales, los
%cuales no son representables de manera precisa en la computadora debido a la
%aritmética finita que esta maneja. Teniendo esto en cuenta es que la manera
%estándar en cómo se calculan los resultados de ciertas operaciones puede que
%no sea la óptima.\\

Lo que queremos es encontrar una forma de calcular la raíz cuadrada. Hay
diferentes maneras de calcular la raíz cuadrada%{link a methods square root} En
este trabajo vamos a tratar de aproximar la inversa de la raíz cuadrada a
traves de diferentes metodos de ceros de funciones.

En este caso para calcular la inversa de la raíz cuadrada
($\displaystyle\frac{1}{\sqrt{\alpha}}$) buscaremos los ceros de dos funciones.

La resolución de estas 2 funciones es equivalente a resolver el problema de la
inversa de la raíz. A partir de que el problema se plantea de esa manera es que
podemos hacer uso de métodos para encontrar ceros de funciones.\\

Las funciones en cuestión son

\begin{displaymath} f(x) = x^2 - \alpha \end{displaymath}

\begin{displaymath} e(x) = \frac{1}{x^2} - \alpha \end{displaymath}

Cada una de estas funciones provee formas de encontrar la raiz de un numero.
Veamos el caso de f(x).

Los métodos elegidos para encontrar las raices de estas funciones son el método
de Newton y el método de la Secante.

\subsubsection{Análisis de las funciones}

Notemos que alpha siempre tiene que ser positivo, sino no se le puede calcular la raiz cuadrada a un numero negativo. Mas aún alhpa > 0 ya que sino estaríamos diviendo por cero.

\subsubsubsection{f(x)}

primero demostrar que cuando f(x) == 0 -> x == sqrt(alpha)
entonces una vez encontradas las raices podemos hacer 1/x para encontrar
1/sqrt(alpha).

en el caso de e(x) == 0 -> x == 1/sqrt(alpha) por lo que no tenemos que hacer
ninguna otra cuenta.

analicemos graficamente las funciones.

% intertart grafico lindo de f(x)

f(x) es una parabola. Al ser alpha > 0 podemos ver que *siempre* tiene dos
raices. Más aún f(x) es simétrica por lo cual podemos encontrar cualquiera de
las dos raices y con esta cambiarle el signo y obtener la otra. De esta forma
no nos preocuparemos por obtener la raiz positiva ya que nos es indistinto que
raiz conseguimos con los metodos.

% insertar grafico lindo de e(x)

e(x) con alpha > 0 tambien tiene siempre dos raices por lo que al igual que con
f(x) nos es indistinto cual de las dos obtenemos. En este casi es importante
notar que en el 0 hay una asíntota de las ordenadas.
 

Los dos métodos que elegimos trabajan o con la tangente de las funciones en un
punto o con una aproximación de esta. Veamos la convavidad de las funciones.o

% a partir de aca es todo chamuyo... ver que dejar y que sacar, porque en
% realida se puede hacer buen analisis analitico pero no se como.. buscar en
% google o wikipedia quizas


% graficos lindos de f''(x) y e''(x).

veamos que f\'\'(x) == 2. esta funcion es convexa. al ser constante y por la
forma que tienen las derivadas asumimos que siempre va a converger.

veamos e\'\'(x) se nos va al sorete en el 0. por algo que no se, que nos tenemos
que sacar de la galera esta no siempre va a converger!!! pero no sabemos bien
por que


Cómo se puede ver los valores que hacen cero a esas funciones son de la forma
$x = \sqrt{\alpha}$ en el primer caso y $x = \frac{1}{\sqrt{\alpha}}$ en el
segundo.

\subsubsectioj{metodos}
podemos ver que se puede utilziar tanto newton como secante porque las
funciones cumplen lo que pida cada una.

podemos utilizar la secante porque .. y tambien newton porque pide todo esto y
que exista la derivada primera de la funcion, las cuales tenemos para e y para
f.

\subsubsection{hipotesis}
creemos que newton va a funcionar mucho mejor que secante porque su orden de
convergencia teorico es mejor.

Por otro lado no estamos seguros de cual de las dos funciones va a ser mejor
porque no podemos interpretar funciones y como afectan las tangentes a los
metodos.

% no se si esto viene aca o mas abajo, quizas mas abajo o arriba, pero si nos
% interesa ver los dos casos, el caso con bajo error y el caso rapido.
Por ultimo queremos hacer un analisis para un calculo aproximado pero MUY
RAPIDO al estilo quake. Para este puede ser que las conclusiones sean
diferentes a casos donde puede haber mas iteraciones.  por que? porque para un
caso podemos empezar mucho ma cerca de la raiz y hacer pocas iteraciones y en
el otro caso podemos empezar muy lejos pero con algunas iteraciones mas ya
estar mucho mas cerca. pero para ej 3 iteraciones es mejor el metodo y funcion
"lento".


% Contendra una breve explicacion de la base teorica que fundamenta los metodos
% involu- crados en el trabajo, junto con los metodos mismos. No deben
% incluirse demostraciones de propiedades ni teoremas, ejemplos innecesarios,
% ni deniciones elementales (como por ejemplo la de matriz simetrica). En vez
% de deniciones basicas es conveniente citar ejemplos de bibliografa
% adecuada.  Una cita vale mas que mil palabras

\newpage \section{Desarrollo}
% Deben explicarse los metodos numericos que utilizaron y su aplicacion al
% problema concreto involucrado en el trabajo practico. Se deben mencionar los
% pasos que si- guieron para implementar los algoritmos, las dicultades que
% fueron encontrando y la descripcion de como las fueron resolviendo. Explicar
% tambien como fueron planteadas y realizadas las mediciones experimentales.
% Los ensayos fallidos, hipotesis y conjeturas equivocadas, experimentos y
% metodos malogrados deben gurar en esta seccion, con una breve explicacion
% de los motivos de estas fallas (en caso de ser conocidas)

\subsection{implementacion}

hicimos una implementacion que nos permite decidir parametricamente lo siguiente:

* funcion e o f
* metodo newton o secante
* criterio de parada
  * iteraciones
  * tiempo
  * error absoluto
  * error relativo
* eleccion de x0 y x1 (x1 solo para la secante)

esto lo desarrollamos en C y luego hicimos unos helpers en Python para poder
crear experimentos y analizar los resultados. Cada corrida nos devuelve todos
los valores de los criterios de parada y las aproximaciones en cada iteracion.

Estos experimentos son replicables y los puede ver en el axexo {detalle de experimentos}

El codigo se puede optimizar MUCHO, sobre todo en las cuantas que realiza el
cuerpo de cada metodo. Ahora hay una implementacion naif que replica el enfoque
analitico pero que a nivel fierro se puede mejorar.

Por otro lado, agregamos mucho overhead con la flexibilidad, pero podemos ver
que esto afecta a los dos metodos y funciones por igual por lo que no deberia
afectar las comparaciones y resultados.

\subsection{Convergencia}

empezamos con x0 fijos y vimos que tanto newton como secante siempre convergian
para f(x). % mentira, hacer alguas corridas, tengan fe Esto es tanto para
alphas muy chicos (cercanos al cero) como para alhpas muy grandes.

en cambio para e(x) vimos que nunca convergia, al menos las aproximaciones
crecian muy rapidamente, tanto que en un puñado de iteraciones superaban el
mayor número representable por punto flotante doble (devolvia inf). % mostrar
alguna corrida donde pase esto.

Entonces para f(x) decidimos utilizar un x0=alpha que siempre funciona. Newton
tiene una convergencia teorica muy rapida por lo que no deberia influir.

En el caso de e(x) es muy interesante notar como las aproximaciones se alternan
entre positivas y negativas cuando no converge. esto se debe a ???.

% todavia no encontramos un x0 y x1 para e(x) para que converja

empiricamente (diseñar y pensar exp) encontramos que si elejimos un x1 y x0
cerca de e(x) converge pero sino no.

un primer acercamiento a x0 que se nos ocurrio fue utilizar 0.0alpha y
0.00alpha. Esto funciona hasta alphas no muy grandes (mas o menos hasta 10^4,
ajustar es numero con un experimento).

% explicar que es DBL EPSILON y que es DBL MIN
Al hacer estas pruebas nos parece que si x0 y x1 son menos que la raiz
converge. Entonce decidimos utilizar DBL_MIN y DBL_MIN * 2 pero estas rompian
las operaciones aritmeticas de punto flotante. Entonce pasamos a utilizar
DBL_EPSILON y encontramos que funciona muy bien tanto para numeros muy chicos
como para numeros muy grandes. El problema que tiene es que requiere de muchas
iteraciones.

Grafiquemos la cantidad de iteraciones necesarias para diferentes alphas (muy
chicos y cercanos al 1 y muy grandes) para DBL_EPSILON. Lo podemos mejorar?
creemos que si, pero quedará para futura investigacion.

Otra cosa que se nos ocurrio es utilizar aproximaciones de f(x) primero hasta
un cierto error grosero y luego continuar refinando con e(x). Pero para que
esto tenga sentido tuvimos que hacer una comparacion entre los metodos. Ver mas
abajo.

\subsection{comparacion entre criterios de convergencia para f(x) y newton}

hacer muchos exp con grafiquitos y decir o refutar que relativo conviene
siempre mas que absoluto. luego el tiempo y la cantidad de iteraciones.

\subsection{comparacion entre f y e, newton}

para los mismos alphas y criterios de convergencia (probamos con iteraciones,
tiempo, error relativo) vemos cual tarda mas iteraciones y cual tarda mas
tiempo.

Podemos ver que .. es mas rapido que .. que es de suponer por la pendiente que
tiene.

\subsection{comparacion entre newton y secante para e(x)}

ver cual es mas rapido
calcular experimentalmente el orden de convergencia, hay que dividir entre
iteraciones, deberia dar un grafico que luego de algunas iteraciones sea
constante en 2 y en phi.

\section{Resultados}
% Deben incluir los resultados de los experimentos, utilizando el formato mas
% adecuado para su presentacion. Deberan especicar claramente a que
% experiencia corresponde cada resultado. No se incluiran aqu corridas de
% maquina. Algo fundamental en su aprendizaje en la materia es la presentacion
% de resultados de forma clara y concisa para el lector

blah blah blah cual es mejor con cual nos quedamos para el caso preciso y con
cual para el caso rapido donde rapido puede ser 1 2 o 3 iteraciones y preciso
cientos.
\section{Discusión}
% Se incluira aqu un analisis de los resultados obtenidos en la seccion
% anterior (se analizara su validez, coherencia, etc.). Deben analizarse como
% materianimo los lostems pedidos en el enunciado. No es aceptable decir que
% \los resultados fueron los esperados", sin hacer clara referencia a la
% teoremasa a la cual se ajustan. Ademas, se deben mencionar los resul- tados
% interesantes y los casos \patologicos" encontrados.

\subsection{Casos intersantes}

primer caso
===========

bin/tp1 0.0000000000000000000000000000000001 e n r 0.000000000000000000000000000001 1 2
se queda saltando entre dos numeros diferentes

con numeros tan chicos de alhpa el punto flotante explota y se pone a hacer cosas raras, analizarlo un poco mas y decir por que parece que hace esto.


segundo caso
============

una corrida, si le agregamos un 0 al error relativo corta por cantidad de iteraciones y no por llegar al error relativo

*********************** bin/tp1 9999 e s r 0.000000000000001 0.09999 0.009999
Buscando 1 / sqrt(9999.000000)
Utilizando e(x)
Utilizando Secante
Criterio de parada error relativo con limite 0.000000
Utilizando x0=0.099990 y x1=0.009999

0.010026267273090912895971982266019040253013372421264648437500000000000000000000000000000000000000000 Iter
0.010000505837327295852179354085365048376843333244323730468750000000000000000000000000000000000000000 Iter
0.010000500015077814705555248053769901162013411521911621093750000000000000000000000000000000000000000 Iter
0.010000500037503143660466697895117249572649598121643066406250000000000000000000000000000000000000000 Iter
0.010000500037503126313231938127046305453404784202575683593750000000000000000000000000000000000000000 Iter

0.010000500037503126313231938127046305453404784202575683593750000000000000000000000000000000000000000 resultado
5 iteraciones
-0.000000000001818989403545856475830078125000000000000000000000000000000000000000000000000000000000000 error absoluto
-0.000000000000000181917132067792432769858681142675088907883197372229722166281362660811282694339752197 error relativo
0.000067999999999999999463970445923166607826715335249900817871093750000000000000000000000000000000000 tiempo

*********************** bin/tp1 9999 e s r 0.0000000000000001 0.09999 0.009999
0.010000500037503124578508462150239211041480302810668945312500000000000000000000000000000000000000000 Iter
0.010000500037503124578508462150239211041480302810668945312500000000000000000000000000000000000000000 Iter
0.010000500037503124578508462150239211041480302810668945312500000000000000000000000000000000000000000 Iter
0.010000500037503124578508462150239211041480302810668945312500000000000000000000000000000000000000000 Iter
0.010000500037503124578508462150239211041480302810668945312500000000000000000000000000000000000000000 Iter
0.010000500037503124578508462150239211041480302810668945312500000000000000000000000000000000000000000 Iter

0.010000500037503124578508462150239211041480302810668945312500000000000000000000000000000000000000000 resultado
1000 iteraciones
0.000000000001818989403545856475830078125000000000000000000000000000000000000000000000000000000000000 error absoluto
0.000000000000000181917132067792432769858681142675088907883197372229722166281362660811282694339752197 error relativo
0.011587999999999999342636947119444812415167689323425292968750000000000000000000000000000000000000000 tiempo


\section{Conclusiones}
% Esta seccion debe contener las conclusiones generales del trabajo. Se deben
% mencionar las relaciones de la discusion sobre las que se tiene certeza,
% junto con comentarios y observaciones generales aplicables a todo el proceso.
% Mencionar tambien posibles extensiones a los metodos, experimentos que hayan
% quedado pendientes, etc

todo depende, de que depende? de si buscas velocidad o precision



\section{Apéndices}
% En el apendice A se incluira el enunciado del TP. En el apendice B se
% incluiran los codigos fuente de las funciones relevantes desde el punto de
% vista numerico. Resultados que valga la pena mencionar en el trabajo pero que
% sean demasiado especicos para aparecer en el cuerpo principal del trabajo
% podran mencionarse en sucesivos apendices rotulados con las letras mayusculas
% del alfabeto romano. Por ejemplo: la demostracion de una propiedad que
% aplican para optimizar el algoritmo que programaron para resolver un
% problema.

\section{Referencias}
% Es importante incluir referencias a libros, articleculos y paginas de
% Internet consultados durante el desarrollo del trabajo, haciendo referencia a
% estos materiales a lo largo del informe. Se deben citar tambien las
% comunicaciones personales con otros grupos

Wikipedia
Burden

\section{Anexos}
\subsection{Pseudocodigo de metodos}
copiar pseudocodigo (comentarios) de newton_start newton_iter secante_start
secante_iter y loop principal

\subsection{Detalle de experimentos}

en este anexo detallamos los diferentes experimentos, como replicarlos y la
motivacion detras de cada uno de ellos.

por cada experimento indicar en que archivo esta, y que corridas se hicieron
con que datos y por que elegimos hacer ese exp.

\end{document}
