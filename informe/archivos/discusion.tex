\section{Discusión y Conclusiones}
% Se incluira aqu un analisis de los resultados obtenidos en la seccion
% anterior (se analizara su validez, coherencia, etc.). Deben analizarse como
% materianimo los lostems pedidos en el enunciado. No es aceptable decir que
% \los resultados fueron los esperados", sin hacer clara referencia a la
% teoremasa a la cual se ajustan. Ademas, se deben mencionar los resul- tados
% interesantes y los casos \patologicos" encontrados.

En esta parte analizaremos, cuestionaremos y concluiremos sobre aquellos puntos que nos parecieron interesantes pero que además fueron preponderantes en el desarrollo del presente trabajo.

\subsection{Elección de los puntos iniciales de las sucesiones} % (fold)
\label{sub:elecci_n_de_los_puntos_iniciales_de_las_sucesiones}

En la secciones \ref{ssub:ajuste_f_x0_newton}, \ref{ssub:ajuste_e_x0_newton}, \ref{ssub:ajuste_e_x0_x1_secante}
% Hablar sobre las variantes que descartamos al hacer este tipo de enfoque y sus posibles repercusiones

% subsection elecci_n_de_los_puntos_iniciales_de_las_sucesiones (end)

\subsection{Análisis de los resultados en función del tiempo y las iteraciones} % (fold)
\label{sub:an_lisis_de_los_resultados_en_funci_n_del_tiempo_y_las_iteraciones}

\subsubsection{Iteraciones} % (fold)
\label{ssub:iteraciones}

Para este gráfico (Figura~\ref{fig:comparacion_iteraciones}) no existe demasiada discusión ya que el resultado cumple con las expectativas que teníamos previamente. Cabe destacar que las pendientes de las rectas con respecto al origen en módulo no son iguales, esto es interesante dado que el análisis previo de cada función nos mostró que podíamos hacer uso indistinto de cualquiera de las raíces, lo cual hacía suponer que el gráfico tendría que tener mayor simetría.\\

Mas allá de estos detalles, pudimos cerciorar empíricamente que si se toma en cuenta la cantidad de iteraciones, usando $f(x)$ con el método de Newton es la opción mas recomendable

% subsubsection iteraciones (end)

\subsubsection{Tiempo de ejecución} % (fold)
\label{ssub:tiempo_de_ejecuci_n}

En esta gráfico (Figura~\ref{fig:comparacion_tiempos}) es en donde pudimos ver un comportamiento al menos no esperado de nuestra parte. Suponemos que nuestra hipótesis no se cumple, pues la derivada en este caso es fácil de calcular y evaluar, haciendo que el factor que menos beneficiaba a este método no afecte para este caso a la performance.

% subsubsection tiempo_de_ejecuci_n (end)

% subsection an_lisis_de_los_resultados_en_funci_n_del_tiempo_y_las_iteraciones (end)

\subsection{Resoluciones finales} % (fold)
\label{sub:resoluciones_finales}

% subsection resoluciones_finales (end)

\subsection{Trabajos futuros} % (fold)
\label{sub:trabajos_futuros}

% subsection trabajos_futuros (end)