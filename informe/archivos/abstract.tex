\section{Abstract}

El objetivo de este trabajo es encontrar la inversa de la raiz cuadrada de un numero de forma eficiente. Para esto analizamos dos funciones diferentes en las cuales los puntos en que se anulan nos permiten encontrar la inversa de la raíz cuadrada.\\

Para encontrar estos puntos utilizamos el método de Newton y el de la Secante. Luego, comparamos los resultados de utilizar estos métodos en cada función ajustando parámetros para buscar la mejor solución.\\

Por último, enunciamos ventajas y desventajas comparativas basados en resultados y análisis teóricos.

% En este trabajo vamos a mostrar dos formas de poder
% calcular la inversa de la raíz cuadrada, a partir de adaptar el problema a la
% búsqueda de ceros de una función. Iremos documentando también los resultados
% obtenidos a partir de ciertas familias de inputs elegidas con cierto
% criterio.\\

% En particular veremos el método de Newton y el de la Secante para llegar al
% resultado mencionado.\\

% Por último, a partir de la observación de los experimentos daremos nuestro juicio sobre las ventajas y desventajas de ambos métodos y nuestra recomendación sobre cuál usar.\\

{\bf Palabras clave:}
\begin{itemize} 
    \item Método de Newton 
    \item Método de la Secante 
    \item Cero de funciones
    \item Raíz cuadrada
\end{itemize}
