\section{Conclusiones}
% Esta seccion debe contener las conclusiones generales del trabajo. Se deben
% mencionar las relaciones de la discusion sobre las que se tiene certeza,
% junto con comentarios y observaciones generales aplicables a todo el proceso.
% Mencionar tambien posibles extensiones a los metodos, experimentos que hayan
% quedado pendientes, etc
todo depende, de que depende? de si buscas velocidad o precision
Veamos lo que podemos implicar a partir de la comparación que se hizo de cada gráfico
\begin{itemize}
\item {\bf Cantidad de iteraciones f(x) con Newton y Secante:} Acá se puede observar que a medida que el $x_0$ crece la cantidad de iteraciones crece pero mas despacio, casi logrando un crecimiento logarítmico. Sin embargo es apreciable que el método de Newton toma menos iteraciones.
\item {\bf Error Absoluto f(x) con Newton y Secante:} En los gráficos se puede apreciar que cuando $x_0$ es chico Newton y Secante alcanzan el criterio de parada casi en los mismos puntos. Pero a medida que $x_0$ crece Secante empieza a llegar antes al criterio de parada. Se podría decir que Secante es mas preciso que Newton.
\item {\bf Tiempo de Computo f(x) con Newton y Secante:} En este caso es claro ver  

\end{itemize}



