\section{Abstract}

En este trabajo vamos a mostrar dos formas de poder
calcular la inversa de la raíz cuadrada, a partir de adaptar el problema a la
búsqueda de ceros de una función. Iremos documentando también los resultados
obtenidos a partir de ciertas familias de inputs elegidas con cierto
criterio.\\

En particular veremos el método de Newton y el de la Secante para llegar al
resultado mencionado\\

Los experimentos nos terminaron mostrando que el método de Newton termina
siendo mejor en cuestiones de orden de convergencia y error.\\

{\bf Palabras clave:}
\begin{itemize} 
    \item Método de Newton 
    \item Método de la Secante 
    \item Criterios de comparación 
\end{itemize}
