\section{Abstract}

En este trabajo vamos a mostrar dos formas de poder
calcular la inversa de la raíz cuadrada, a partir de adaptar el problema a la
búsqueda de ceros de una función. Iremos documentando también los resultados
obtenidos a partir de ciertas familias de inputs elegidas con cierto
criterio.\\

En particular veremos el método de Newton y el de la Secante para llegar al
resultado mencionado.\\

Por último, a partir de la observación de los experimentos daremos nuestro juicio sobre las ventajas y desventajas de ambos métodos y nuestra recomendación sobre cuál usar.\\

{\bf Palabras clave:}
\begin{itemize} 
    \item Método de Newton 
    \item Método de la Secante 
    \item Ceros de funciones
\end{itemize}
