\section{Anexos}
% \subsection{Pseudocodigo de metodos}
% copiar pseudocodigo (comentarios) de %newton_start newton_iter secante_start
% %secante_iter y loop principal

% \subsection{Detalle de experimentos}

% en este anexo detallamos los diferentes experimentos, como replicarlos y la
% motivacion detras de cada uno de ellos.

% por cada experimento indicar en que archivo esta, y que corridas se hicieron
% con que datos y por que elegimos hacer ese exp.
\begin{center}
\begin{tabular}{r|cr}
 \begin{tabular}{c}
{\large\bf\textsf{\ M\'etodos Num\'ericos\ }}\\ 
Segundo Cuatrimestre 2013\\
{\bf Trabajo Pr\'actico 1}
\end{tabular} &
% \begin{tabular}{@{} p{1.6cm} @{}}
% \includegraphics[width=1.6cm]{../../logodpt.jpg}
% \end{tabular} &
\begin{tabular}{l @{}}
 \emph{Departamento de Computaci\'on} \\
 \emph{Facultad de Ciencias Exactas y Naturales} \\
 \emph{Universidad de Buenos Aires} \\
\end{tabular} 
\end{tabular}
\end{center}

\vskip 25pt
\hrule
\vskip 11pt
 
\textbf{Introducci\'on}

En las \'ultimas dos d\'ecadas se han producido avances muy significativos en el \'area de Computaci\'on Gr\'afica, en
particular en el desarrollo de animaciones y video juegos en 3D donde se obtienen resultados muy detallados y con un
alto nivel de realismo. 

Un aspecto importante que contribuye en este sentido es la iluminaci\'on de la escena y el reflejo de la luz en los
objetos o personajes de la misma. A grandes rasgos, el manejo de la iluminaci\'on se hace de la siguiente forma. Dada
una superficie en el espacio, se calculan los vectores normales a la misma (recordar \emph{An\'alisis II}) sobre un
conjunto determinado de puntos y luego estos vectores son utilizados, en conjunto con el modelo de iluminaci\'on, para
calcular su color final y la interacci\'on con otras superficies. Adem\'as, por cuestiones pr\'acticas, los vectores
normales deber ser almacenados como vectores unitarios, es decir, que su norma Euclideana ($\|~\|_2$) sea 1.

Dado un vector $y \in \mathbb{R}^3$ cualquiera, podemos convertirlo en uno unitario dividi\'endolo por $\|y\|_2$, es
decir, 
\begin{eqnarray*}
\| y \|_2 & = & \sqrt{y_1^2 + y_2^2 + y_3^2}\\
z & = & \frac{y}{\|y\|_2}.
\end{eqnarray*}
\noindent Durante la ejecuci\'on del programa, esta operaci\'on es realizada millones de veces por segundo, por lo cual
es importante realizarla en el menor tiempo posible, eventualmente resignando precisi\'on en el
resultado. Dado el contexto, peque\~nas reducciones en el tiempo de ejecuci\'on pueden mejorar considerablemente el
comportamiento general.

\textbf{El problema}

Sean $y = (y_1, y_2, y_3) \in \mathbb{R}^3$ un vector gen\'erico y $\alpha = y_1^2 + y_2^2 + y_3^2$. El problema de
obtener un vector unitario que posea la misma direcci\'on que $y$ lo podemos plantear como multiplicarlo por
$1/\sqrt{\alpha}$. Luego, el problema de normalizar un vector radica principalmente en el c\'alculo de este valor, que
involucra una divisi\'on y el c\'alculo de una ra\'iz cuadrada. 

Sin utilizar funciones ya provistas por el lenguaje de programaci\'on a utilizar, el c\'alculo de $1/\sqrt{\alpha}$ se
pueden formular como un problema de b\'usqueda de ceros de una funci\'on de (al menos) las siguientes dos formas:
\begin{itemize}
\item Aproximar $\beta = \sqrt{\alpha}$ como un cero de $f(x) = x^2 - \alpha$, y luego realizar $1/\beta$. 
\item Definir la funci\'on $e(x) = \frac{1}{x^2} - \alpha$, que permite calcular el error de una aproximaci\'on de
$1/\sqrt{\alpha}$. En particular, uno de los ceros de esta funci\'on es el valor buscado.
\end{itemize}

Estas dos reformulaciones del problema nos permiten atacarlo con m\'etodos de b\'usqueda de ceros de funciones en una
variable.

\textbf{Enunciado}

El objetivo del trabajo pr\'actico consiste en implementar un programa que permita calcular, dado $\alpha \in
\mathbb{R}$, $1/\sqrt{\alpha}$. Para ello, se deber\'a considerar las funciones $f(x)$ y $e(x)$ definidas
anteriormente, distintos m\'etodos vistos en clase que permitan resolver el problema planteado y realizar un an\'alisis
completo del comportamiento de los mismos. 

Los requisitos m\'inimos a cumplir son los siguientes:

\begin{itemize}
\item Implementar el m\'etodo de Newton para la funci\'on $f(x)$. Incluir en el informe la demostraci\'on de
convergencia (Ejercicio 3, Pr\'actica 1). Para la funci\'on $e(x)$, implementar al menos dos m\'etodos (uno de los
cuales debe ser el de Newton).   
\item Para cada m\'etodo, estudiar experimentalmente la convergencia, tiempo de ejecuci\'on, cantidad de iteraciones,
criterios de parada, precisi\'on en el resultado, y cualquier otro par\'ametro que considere necesario evaluar. Realizar experimentos
computacionales considerando un rango amplio de valores posibles para $\alpha$ y distintos puntos iniciales
para los m\'etodos. Analizar y justificar detalladamente los resultados obtenidos.
\item Una vez fijados los mejores par\'ametros para cada m\'etodo, realizar una comparaci\'on entre las tres formas
alternativas de resolver el problema (Newton para $f(x)$, y Newton m\'as el otro m\'etodo para $e(x)$) en t\'erminos de
tiempo de ejecuci\'on, precisi\'on en la soluci\'on, cantidad de iteraciones, etc. Determinar experimentalmente que
variante seleccionar\'ia para su utilizaci\'on en la pr\'actica.
\end{itemize}
