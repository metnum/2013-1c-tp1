\section{Abstract}

El objetivo de este trabajo es encontrar la inversa de la raíz cuadrada de un
número de forma eficiente. Para esto analizamos dos funciones diferentes en las
cuales los puntos en que se anulan nos permiten encontrar la inversa de la raíz
cuadrada.

Para encontrar estos puntos utilizamos el método de Newton y el de la Secante.
Luego, comparamos los resultados de utilizar estos métodos en cada función
ajustando parámetros para buscar la mejor solución.

Por último, enunciamos ventajas y desventajas comparativas basados en
resultados y análisis teóricos.

{\bf Palabras clave:}
\begin{itemize}
    \item Método de Newton
    \item Método de la Secante
    \item Cero de funciones
    \item Raíz cuadrada
\end{itemize}
